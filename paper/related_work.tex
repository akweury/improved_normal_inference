

In order to estimate normals of an object surface.

In 2012, Holzer et al. \cite{Holzer.S} presented a read-time method, which is able to run algorithm in a high frame speed. They smooth the depth data in order to handle the noise of depth image. The speed is accelerated via integral image. The drawbacks are, as mentioned in the paper, the normals error go up when point depths change severely.

In 2018, Yu et al. \cite{gconv} presented a CNN based method with guided 

In 2019, Ben-Shabat et al. \cite{Ben-Shabat_2019_CVPR} presented a CNN based method.


In 2021, Zhou et al.  \cite{zhou2021fast}


\section{Sparse Input processing}
%% gconv, nconv, median filter...
The depth-map captured by active sensors are usually full with missing pixels and holes. Thus a preprocessing for the depth map is necessary before it is fed into a neural network.

Generally, it can be solved as image inpainting problems.\cite{inpainting1},\cite{inpainting2}. Recently, some deep learning based method for image inpainting achieved quite good performance for the hole mending task. 
Notably, in 2016, Oord et al. \cite{gated_activation} proposed a gated activation unit for a CNN model,
\[\textbf{y} = \tanh (W_{k,f} * \textbf{x}) \odot \sigma (W_{k,g} * \textbf{x})\]
to substitute the standard activation layer, where $ \sigma $ is the sigmoid function, which constricts the output value of second part between $ [0,1] $.  The function is inspired by Long Short-Term Memory (LSTM) \cite{lstm} and Rated Recurrent Unit (GRU).\cite{gru} It is originally used for learning complex interactions as LSTM gates does. In 2018, Yu et al. \cite{gconv} employed same function for free-form image inpainting, which can be used to learn mask automatically from image it self.

Different to aforementioned approaches, Knutsson et al. in 2005 introduced normalized convolution \cite{nconv} dealing with missing sample case for convolution operation. 
In 2018, Eldesokey et al. \cite{ncnn} applied normalized convolution in CNN as normalized convolution layer that takes both sparse depth map and a binary confidence map as input to perform scene depth completion.  
In 2020, Eldesokey et al. \cite{pncnn} focus on modeling the uncertainty of depth data instead of assuming binary input confidence.

\section{Normal Inference}
